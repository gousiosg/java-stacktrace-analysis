\documentclass[conference]{IEEEtran}

\usepackage[pdftex]{graphicx}
\usepackage[cmex10]{amsmath}
\usepackage{algorithmic}
\usepackage{caption}
\usepackage{subfig}
\usepackage{url}
\usepackage{balance}

\begin{document}
\title{What Can Stack Traces Tell us about the Use of Exceptions in Java?}

\author{\IEEEauthorblockN{Roberta Coelho\IEEEauthorrefmark{1},
Georgios Gousios\IEEEauthorrefmark{2}, Arie van Deursen\IEEEauthorrefmark{2},
Lucas Mariano\IEEEauthorrefmark{1}}
\IEEEauthorblockA{\IEEEauthorrefmark{1}Federal University of Rio Grande do Norte\\
Natal, Brazil\\
Email: \texttt{\{roberta, lucasm\}@dimap.ufrn.br}}
\IEEEauthorblockA{\IEEEauthorrefmark{2}Delft University of Technology\\
Delft, The Netherlands\\
Email: \texttt{\{g.gousios,arie.vandeursen\}@tudelft.nl}}
}

\newcommand{\todo}[1]{\textbf{TODO}\footnote{\textbf{TODO:} #1}}

% make the title area
\maketitle

\begin{abstract}

In this work, we perform the first large scale analysis of Java stack traces and
pinpoint how they can reveal bad programming practices. We mined and analyzed
the stack traces embedded in all issues of Java projects available GitHub.
Overall, our research set includes 356,057 issues define at 16,836 projects
(from which 28,800 stack traces were extracted). From this set, the stack traces
of 482 Android projects were investigated in more detail, in combination with
source code and bytecode analysis. In this study some patterns of exception
(mis)-use were consistently detected such as: unexpected wrappings (e.g., Errors
being wrapped in checked exceptions) - revealing that Java hybrid exception
model is not fully used according to its purpose;  undocumented runtime
exceptions signaled by third party code - which makes almost impossible for
library clients to protect against such exceptions; and a high prevalence of
java.lang exceptions reported on issues - representing approximately 50\% of the
analyzed issues.


\end{abstract}

%\IEEEpeerreviewmaketitle

\section{Introduction}

Modern applications have to cope with an increasing number of abnormal
computation states that arise as a consequence of faults in the application
itself (e.g., access of null references), noisy user inputs or faults in
underlying middleware or hardware. The exception handling
mechanism~\cite{goodenough1975exception} is one of the most used schemes for
detecting and recovering from such exceptional conditions. Although the
exception handling mechanism have been embedded in several programming languages
(e.g. Java, C++, C\#) and the target of several studies
(e.g.~\cite{miller1997issues,robillard2000designing,shah2010understanding,
garcia2007extracting,garcia2001comparative,cabral2007exception,coelho2011unveiling}),
the exception handling code is often generally poorly understood and the least
tested part of software systems.

Often some languages constructions may lead the developers into believing that
by just re-throwing the exceptions they can forget about the exceptional
situations during the development of the "happy path". This "ignore-for-now"
approach may turn the exception handling into a generalized "goto"
mechanism~\cite{mandrioli1992advances} making the program more complex and even
less reliable. This behaviour may lead to the well known uncaught exceptions
problem~\cite{jo2004uncaught}, one of the main causes of application crashes.

Moreover, some languages use exceptions to represent coding errors
(out-of-bounds array index, division-by-zero, access to a null). In these cases
exceptions are implicitly signaled by the runtime environment when such
condition happens. One of such implicit exceptions (caused by data conversion
from a 64-bit floating point to a 16-bit signed integer) was the responsible for
the famous failure on Ariane 5's first test flight~\cite{lions1996ariane} ---
leading to the rocket self-destructing 37 seconds after launch and a loss of 500
million dollars. Besides Ariane 5, several applications crash everyday due to
uncaught exceptions~\cite{jo2004uncaught}. The consequences of not careful
exception usage may have the opposite effect of its initial intention of
improving system robustness.

In Java, when the program fails due to an uncaught exception, it automatically
terminates, while the system prints a stack trace to the console. A stack trace
includes a detail message and the execution stack frame when the exception
occurred. A typical Java stack trace consists of an ordered list of methods that
were active on the call stack before the exception has occurred. Stack traces
are a useful source of information about system crashes, and is often used to
support developers in debugging~\cite{schroter2010stack}. Moreover, information
from stack traces combined with crash and bug reports can be used to support bug
classification and clustering~\cite{wang2013improving, kim2011crash,
dhaliwal2011classifying}, fault-proneness prediction
models~\cite{kim2013predicting} and even automated bug
fixing~\cite{sinha2009fault} tools.

However, all these works focus on the analysis of a single system at a time. In
our current context, on which we have plenty access to the information available
on open-source repositories like GitHub initial questions arise: What is the
prevalence of exception stack traces on reported issues? What are the common
characteristics of such stack traces?

%What the stack traces available in these repositories can tell us about the use
%of exceptions in Java programs?

%To guide the our study we also compile general guidelines on how to use
%exceptions proposed by Gosling~\cite{gosling2000java},
%Wirfs-Brock~\cite{wirfs2006toward} and Bloch~\cite{bloch2008effective} and then
%we investigate: what stack traces can tell us about the adherence to such
%practices in Java programs?

To answer these questions, we conduct an exploratory study of stack traces in
the wild. Using a custom tool called ExceptionMiner, that was specifically
developed to for this study, we mined stack traces from issues across all Java
projects on GitHub. Overall, we analyzed 356,057 issues from 16,836 projects,
from which 28,800 stack traces were extracted and used as a source of
information for our analysis. The stack trace analysis  was augmented by
additional information extracted using byte code and source code analyses done
on a carefully selected subset of our initial sample (482 Android projects).
%This study was divided in two phases: in the fist phase all the issues of all
%Java projects available on GitHub were analyzed and the embedded stack traces
%extracted using ExceptionMiner.

To guide this exploration we compiled general guidelines on how to use
exceptions proposed by Gosling~\cite{gosling2000java},
Wirfs-Brock~\cite{wirfs2006toward} and Bloch~\cite{bloch2008effective} and then
we investigate what stack traces can tell us about the adherence to such
practices in Java programs.
%~\todo{This should be our motivation, as this is
%our biggest contribution. I already changed the text. Lets discuss about this
%new version}


%GGG: I think this is too much detail for the introduction In the second phase
%of this study, we combined the information of the mined stack traces with
%source code and bytecode analysis. To do so we selected the subset of 482
%Android repositories. The Android repositories were chosen because this kind of
%application is known for dealing with several sources of exceptions, as they
%are usually io-intensive (i.e., deals with Network, Bluetooth, camera, and
%other resources), and multi-htreaded. Moreover, they are based on a single
%platform which allowed more fine-grained analysis of their exception behaviour.

This study is built on the following assumption: the stack traces reported on
issues contain relevant information related to system crashes. Several studies
and techniques have also based on this assumption
(e.g.,~\cite{sinha2009fault}~\cite{dhaliwal2011classifying}~\cite{kim2013predicting})
and we build on it as well. Some outcomes were consistently detected through
this large scale analysis of exception stack traces in the wild, such as:

\begin{itemize}

  \item  A multitude undocumented runtime exceptions signaled by third party
    code (i.e., libraries and Android platform).

  \item  Odd exception chains (e.g., Error wrapping checked and runtime
    exceptions and vice-versa). As an indication that the purpose of Java hybrid
    exception model may not have been adequately used.

  \item  Stack traces caused by uncaught checked exceptions.

\end{itemize}

Hence, the contributions of this study are as follows:
\begin{itemize}

  \item  It performs the first large scale analysis of Java stack traces and how
    they can reveal bad-practices on the use of exceptions in Java.

  \item  It briefly introduces ExceptionMiner, a tool developed to support the
    analysis.

\end{itemize}


The contributions of this work allow for developers of robust Java applications
to familiarize with the most common reported exceptions mis-uses which is the
first step to help developers to avoid making them; (ii) designers of languages
to consider ways of reducing the abundance of NullPointerExceptions, (iii) and
tool designers to consider developing tools that enable a cross-project defect
analysis. We strongly believe that, specially in the open-source environment,
faults are not to be hidden in a private bug issue. Faults should be share and
discussed, so that a developer can learn from other projects mistakes.
Currently, the search facilities of repositories are very limited, and the
ExceptionMiner tool is a contribution in this direction.

%GGG: Personally, I hate this paragraph in all papers as it usually reads like a space-filler
The remainder of this paper is organized as follows. Section 2 presents a
background on error handling mechanisms and Java exception handling constructs
and practices. Section 3 defines research questions, the configuration of our
exploratory study, the target applications and the evaluation procedures.
Section 4 reports our analysis of the empirical data collected relating to the
first research questions. Section X is dedicated to provide insights for the 3rd
research question. Section 6 provides further discussions and lessons learned
and discusses the threats to validity associated to this study. Finally Section
7 describes the related work, and Section 8 presents our conclusions and
directions for future work.


\section{Background}

\subsection{Error handling in programming languages}

Techniques for error detection and handling are no more an optional add-on but a
fundamental part of the system~\cite{bruntink2006discovering}. Several
approaches exists for detecting errors during system execution and responding to
them accordingly, such as the return-code idiom (very popular in C programs and
operating systems), and the exception handling mechanism, embedded in many main
stream programming languages.

The main disadvantage of return codes is the fact that the programmer can just
ignore the return code and let the program execute in inconsistent state. The
exception mechanism~\cite{goodenough1975exception} was proposed as a way of
preventing errors from being ignored, and was embedded in many mainstream
programming languages. Among some similarities they mainly differ in the way
exceptions are represented and handled in each
language~\cite{garcia2001comparative}:

\noindent\emph{Checked Approach}. In languages such Modula-3, Guide, and
Extended Ada all exceptions are checked, which means that the compiler
statically checks if appropriate handlers are provided within the system.
Accordingly, the programmer must explicitly specify every exception that can be
signaled to support the compile time checking. This approach is known to be
costly to maintain (Dooren and Steegmans, 2005) since exception updates are
cascading; if a new exception type is added to a method signature then all other
methods directly using this method must either change their signatures to
include the new exception type or handle the new exception.

\noindent\emph{Unchecked Approach}. In languages such as Lore, C++, and Arche
all exceptions are unchecked but the developer can optionally list exception
types on method signatures. However, the developer is not warned by the compiler
if an unchecked exception is not handled inside the application. Such an
approach may hinder robustness because the client of a method cannot easily know
which unchecked exceptions may be thrown, unless the code of the method and the
methods called from it is inspected, which can be a very time consuming or
infeasible task.



\noindent\emph{Hybrid Approach}. In Java a method can throw either checked or
unchecked exceptions. Java is the only language that allows this hybrid
approach.

\subsection{Exception handling in Java} \label{sec:extypes}

In Java, an exception is represented according to the class hierarchy shown in
Figure~\ref{fig:exchier}.  According to this structure every exception is an
instance of the Throwable class, and can be of three kinds: checked exceptions
(extends Exception), run-time exceptions (extends RuntimeException), and errors
(extends Error)~\cite{gosling2000java}.

\begin{figure} \centering \includegraphics[width=\hsize]{new2_hierarchy.png}
  \caption{Exception Hierarchy in Java} \label{fig:exchier} \end{figure}

Neither run-time exceptions nor errors need to be specified on method
signatures, and hence both are referred as unchecked exceptions. The Java
Virtual Machine represents as runtime exceptions invalid operations detected in
the program (out-of-bounds array index, divide-by-zero error, null pointer
references) from which most of the programs are not expected to recover. By
convention an Error represents an unrecoverable condition which usually results
from failures detected by the Java Virtual Machine, such as OutOfMemoryError and
normally should not be handled inside the application.

The Java specification~\cite{gosling2000java} suggests that user-defined
exceptions should be checked exceptions. The main reason is because doing so the
callers of a method will know about the exceptions that a method can throw, and
so they can decide what to do about them. However, there is a long-lasting
debate about the use of checked and unchecked exceptions in
Java~\cite{javatut,stackoverlow,debate}\footnote{152 questions in Stackoverlow
are related to this debate}, since there are pros and cons associated to each of
them.

\subsubsection{Exception Throwing and Handling}

In Java an exception can be signaled in one of the following
circumstances~\cite{gosling2000java}:

\begin{itemize} \item  explicitly thrown when a throw statement is executed;
  \item  implicitly thrown by the JVM when the evaluation of an expression
    violates the normal semantics of language (e.g., out-of-bounds array index,
    division-by-zero, access to a null reference); or \item implicitly thrown by
    the JVM due to an internal error or resource limitation (e.g.,
    OutofMemoryError).  \end{itemize}
%\todo{This sub-section is quite similar to the section introduction, shouldn't
%we merge them?}
When an exception is signaled it causes a transfer of control from the point
where the exception occurred to a point that can be specified by the programmer
(exception handler). In Java it is represented by the try-catch block and is
responsible for performing any recovery action necessary to bring the software
system back to a normal state and, whenever this is not possible, to log the
exception and abort the system in an expected safe way.

In Java, as in many languages, the search for the handler to deal with a raised
exception occurs along the dynamic invocation chain. This is claimed to increase
the reusability, since the invoker of an operation can handle it in a wider
context ~\cite{goodenough1975exception}. If no handler is found in the chain,
the default handler associated to the Thread or ThreadGroup is invoked, which
usually terminates the system and prints the exception stack trace.

\subsubsection{Exception Chaining} Exception Chaining is a common way of
propagating exceptions across components of different abstraction levels. It is
implemented by special exception constructor which receives another exception as
a parameter. An example a multiple exception chaining is depicted on the stack
trace in Figure~\ref{fig:wrapping}.
%This stack trace was extracted from an issue of an Android application in
%Github.

The bottom part of the stack trace is the \emph{root cause}, which indicates the
first reason for the error to be thrown (in this case, the computer run out of
memory). The top part of the stack trace indicates the location of the exception
manifestation (which will call \emph{top level exception} along this paper). The
execution flow  between the root cause and the exception manifestation may
include other intermediate exception wrappings. In all levels, the exception
\emph{signaler}, is the method that threw the exception, represented on the
stack trace as the first method call below the exception declaration.

\begin{figure} \centering \includegraphics[scale=0.6]{wrappings}
\caption{Example of an Exception stack trace in Java.}
\label{fig:wrapping}
\end{figure}

\subsection{Exception Handling Best Practices}
\label{sec:best}

%\todo{This section feels a bit disconnected.}
Several general guidelines have been proposed on how to use
exceptions~\cite{mandrioli1992advances,gosling2000java,wirfs2006toward,
bloch2008effective}. To guide our investigation, we compile general guidelines
related to exception types and the propagation of exception in Java programs
\footnote{We also compiled guidelines related to exception handling but they are
out of the scope of this paper, since they do not relate to information
available on stack traces.}.

\noindent\emph{Meaning of Exception Types}

\textbf{1) Use checked exceptions to represent recoverable
conditions}(~\cite{mandrioli1992advances,gosling2000java,wirfs2006toward,bloch2008effective}).
The developer should use checked exceptions for conditions from which the caller
is expected to recover. By confronting the API user with a checked exception,
the API designer is telling the client to handle the exceptional condition. The
client can explicitly ignore the exception (swallowing, or converting it to
other type) at the expense of the program's robustness~\cite{gosling2000java}.

\textbf{2) Error represents an unrecoverable condition detected by the JVM which
should not he handled}~\cite{gosling2000java}. It results from failures detected
by the Java Virtual Machine which indicate resource deficiencies, invariant
failures or other conditions that make impossible the program to recover.

\noindent\emph{Exception Throwing}

\textbf{3) A method should throw exceptions that precisely define the
exceptional condition}~\cite{gosling2000java,bloch2008effective}. To do so,
developers should either try to reuse the exception types already defined in the
JVM or they should create a specific exception. Throwing general types such as a
pure Exception or a RuntimeException is hence considered a bad practice.

\textbf{4) Do not repeatedly re-throw exceptions, handle exceptions as close as
possible to the problem}~\cite{wirfs2006toward}. Programs that frequently throw
exceptions lose in performance ~\cite{wirfs2006toward,gosling2000java}.
Moreover, close to the place that the exception was throw the caller usually has
enough contextual information to perform  corrective actions. When the exception
is propagated far away from the method where it was signaled (i.e. further away
from the problem) it can be difficult to take meaningful recovery actions.

\noindent\emph{Exception Documentation}

\textbf{5) Document all exceptions thrown by a
method}~\cite{mandrioli1992advances,gosling2000java,wirfs2006toward,bloch2008effective}).
The exceptions thrown by a method are an important part of methods interface,
and is required to use the method properly. The checked exceptions are already
part of the  methods signature, and the method caller is aware of the checked
exceptions being thrown by it. According to~\cite{bloch2008effective}, it is
also wise to document the explicitly signaled runtime exceptions\footnote{This
excludes the implicitly signaled JVM runtime exceptions related to coding
errors} as carefully as checked exceptions. Doing so, the clients of a method
will be aware of the exceptions the method can throw. If the developer fails to
do follow this practice (specially when developing library code) it will be
difficult or even impossible for the caller to make effective use of such method
~\cite{wirfs2006toward, bloch2008effective}.

These best practices guided our exploratory study described in the next
sections. Based on information extracted from stack traces (complemented by
bytecode and source-code analysis), we investigate whether stack characteristics
can reveal whether these practices have been obeyed. Next section discusses the
study procedure and the practices' misuses that emerged from this study.

\section{Study Design}

This work describes an exploratory study which was based on a sequential
mixed-methods approach~\cite{ivankova2006using} for collecting, analyzing, and
integrating both quantitative and qualitative analysis along the stages of the
research process. The main goal of this study was to gain a better understanding
of the use of exceptions in Java programs based mainly on the information
available on exception stack traces. More specifically, we aimed at answering
the following research questions:

\begin{description}

  \item[RQ1] What is the prevalence of stack traces on  reported issues?
    %\item[RQ2] What are the most common exceptions?

  \item[RQ2] What are the common characteristics of such stack traces?

  \item[RQ3] What patterns of exception (mis)-use emerge from the stack trace
    analysis?
    %~\todo{Perhaps rephrase this to: Are the identified best practices followed
    %in practice?}

\end{description}

For RQ1 and RQ2, we explore the domain quantitatively, and use theirs answers as
a basis to investigate the third research question. RQ3 aimed at finding whether
of exception (mis)-uses could be revealed through stack trace analysis. To
support this investigation, source code and bytecode analysis was used to
leverage the  understanding of stack traces. For RQ3 we investigate
quantitatively and highlight interesting cases by exploring cases qualitatively.
Figure~\ref{fig:overviewfig} presents an overview of the mixed-methods approach
conducted to answer these questions.

\begin{figure*}
\centering
\includegraphics[width=\hsize]{overview.pdf}
\caption{Overview of the study based on mixed approaches.}
\label{fig:overviewfig}
\end{figure*}

We begun our investigation using the dataset provided by the GHTorrent
project~\cite{Gousi13}, an off-line mirror of the data offered through the
Github API. The Github API data come in two forms; a streaming data flow lists
events, such as reporting an issue, happening on repositories in real time,
while a static view contains the current state of entities. The data is stored
in unprocessed format, in a MongoDB database, while metadata is extracted and
stored in a relational database. The GHTorrent dataset covers a broad range of
development activities on Github, including issues. The project has been
collecting data since February 2012. Up to Dec 2013, when we queried it to
retrieve our bootstrapping sample which included: 356.057 issues from 16.836
Java projects (step 1).\footnote{A Java project is specified in GitHub with a
specific label.}

After obtaining the bootstrap issue sample from GHTorrent, each issue was
processed by ExceptionMiner tool developed in this work which extracted every
exception defined on issues or on issue comments (step 2). Since this initial
set of Java project is too diverse (it includes libraries, web-based or
stand-alone applications), in the second phase of this study we restrict the
analysis domain to Android projects (step 3). This decision was driven by the
fact that most open source Android applications are known~\cite{ref} for dealing
with several sources of exceptions due to multi-threaded execution and
interaction with input/output resources; moreover they share a common
well-defined platform and a limited set of libraries.

To support a deeper investigation of the stack traces of Android apps, every
exception defined on a stack trace was then classified according to its type
(e.g. Error, Exception or RuntimeException) (step 4). The exception classifier
was initialized with all the exceptions defined on the Java SDK version 7, the
exceptions of the Android (Platform and OS) (level 19), and all the exceptions
of the 482 Android projects selected for this study, extracted in the previous
step. Moreover, all the packages associated to exceptions and signalers were
also extracted to support further analysis (step 5). The signaler classifier is
initialized with information about all packages found on Java SDK version 7, and
Android Platform and OS (level 19). Finally, additional meta-data is combined to
each stack trace (step 6) which used together with manual inspections (step 7)
enabled to provide answers to RQ3.

\subsection{Github Data}
\label{sec:git}

\noindent\emph{Kinds of Analyzed Issues.} Issues on Github are different from
issues on dedicated bug tracking tools such as Bugzilla and Jira. The most
important difference is that there are no predefined fields for bug reporters to
select from (e.g. severity and priority); Github uses instead a more open ended
tagging system. Repositories are offered a pre-defined set of labels, but
repository owners can modify them at will. An issue can have an arbitrary set of
labels attached. Hence, this study does not restrict the analysis of stack
traces defined on defect-issues, based on the assumption that: regardless the
type of issue where its is defined the stack trace information contain relevant
information concerning the exception use. Moreover, issue and pull requests are
dual on Github; all pull requests have a corresponding ``backing'' issue - which
are automatically generated. Therefore, we needed to exclude pull request based
issues from our analysis. Finally, the Github API allows the automated
generation of issues for specific repositories, which automated tools often use
to report crashes. In some cases, this led to high number of issues that
included stack traces. We identified and filtered those cases out (e.g.,
pullWifi project was responsible for almost 50\% of all issues created on
GitHub, it was removed from our analysis and also from GitHub).

\subsection{Android Apps Selection}
\label{sec:android}

%In the second phase of this study we selected a subset of Java applications to
%be analyzed in more detail.The subset chosen was comprised by the Android apps
%included on GitHub (until 23 Feb 2013).
To identify Android projects, we performed a case insensitive search for the
term \textsf{android} in the repository's names and short descriptions.  The
heuristic filtered 2.542 repositories, from which 589 apps had at least one
issue containing a stack trace.

Then we performed further cleanup, inspecting the site of every Android
reporting at least one stack trace, to make sure that they represented real
mobile apps. During this clean up 106 apps were removed because they were either
example projects (i.e., toy projects) or tools to support Android development
(e.g. selendroid, roboeletric -- tools to support the testing of Android apps).
The filtered set consisted of 482 apps, from which approximately 50\% are also
hosted in Google Market Place. For each one of the analyzed projects, we
downloaded the source code and using custom scripts, we extracted their package
hierarchy (recursive list of of Java packages) and their custom exception types
as described next.

\subsection{The ExceptionMiner Tool}
\label{sec:exceptionminer}

To extract exceptions from issues, we implemented ExceptionMiner, a modular
mining tool able to connect to various repositories (such as Google Code,
GHTorrent or even directly to Bugzilla), extract issues, mine stack traces from
them and classify them in predefined categories. The main components of
ExceptionMiner are as follows:

\noindent\emph{StackTraceMiner and Distiller} The first step in the process is
mining references to exceptions and stack traces embedded in issues, distilling
the information that composes a stack trace and storing the results in a
relational database. The extracted stack trace information  the exception
signaler, the root cause, the exception wrappers. The tool is based on a
combination of a regular expression based parser with heuristics able to
identify and filter exception names and stack traces inline with text. In
contrast to existing issue parsing solutions such as Infozilla, the parser
created in this work can extract all causes related to an exception on a stack,
and stack traces embedded in logs files.

%\begin{tiny}
%\begin{verbatim}

%03-01 W/dalvikvm(7924): threadid=1: thread exiting with uncaught exception (group=0x40adf210)
%03-01 15:55:01.609 (7924): FATAL EXCEPTION: main
%03-01  (7924): java.lang.RuntimeException: ...
%E/AndroidRuntime( 7924):  at android....performLaunchActivity(ActivityThread.java:1967)
%03-01 15:55:01.609 (7924):  at android....handleLaunchActivity(ActivityThread.java:1992)
%03-01 15:55:01.609:  at android.app.ActivityThread.access\$600(ActivityThread.java:127)
%E/AndroidRuntime( 7924):  at android....handleMessage(ActivityThread.java:1158)
%\end{verbatim}
%\end{tiny}

\noindent\emph{Exception Classifier} The next step in the ExceptionMiner process
is classifying exceptions as either checked exceptions, runtime exceptions or an
errors. The exception classifier is based on bytecode analysis (based on Design
Wizard~\cite{}) that walks up the type hierarchy of a Java exception until it
reaches a base exception type.

\noindent\emph{Exception Signaler Classifier} After its initialization it
classifies each signaler according to its origin (i.e. Android Plaform,
Aplication, Library, Libcore, or Java) based on pattern matching between the
signaler name and the packages associated to each category.

%\section{Study Operation and Results}
Next section presents the results for each of the study phases, providing both a
quantitative and qualitative analysis of the outcomes.

%\subsection{Phase 1: Analyzing Pure Stack Trace Information of Java Programs}

\section{RQ1: What is the prevalence of stack traces on reported issues?}

Prevalence or prevalence proportion, a term often used in epidemiology,
represents the proportion of a population found to have a particular condition
(e.g. a disease) over the total number of people studied. In our work, we used
this term to evaluate a condition (i.e., presence of a stack trace) over issues
reported on GitHub repositories.

% latex table generated in R 2.15.3 by xtable 1.7-1 package
% Mon Apr 28 15:44:10 2014
\begin{table*}[ht]
\centering
\begin{tabular}{rp{27em}rrrrc}
  \hline
Feature & Description & quant\_5 & mean & median & quant\_95 & histogram \\ 
  \hline
all & All issues recorded for the project. & 1.00 & 23.37 & 3.00 & 80.00 & \includegraphics[scale = 0.1, clip = true, trim= 50px 60px 50px 60px]{hist-31b77719e3773ad7ac2073785de62a5f.pdf} \\ 
  with\_exception & All issues featuring an exception. & 0.00 & 3.21 & 0.00 & 6.00 & \includegraphics[scale = 0.1, clip = true, trim= 50px 60px 50px 60px]{hist-0c1305fc40f695dbdd1f4696db24f147.pdf} \\ 
  with\_stack & All issues featuring a stack trace. & 0.00 & 2.67 & 0.00 & 4.00 & \includegraphics[scale = 0.1, clip = true, trim= 50px 60px 50px 60px]{hist-896524802e915aa3f01ef1ecd2489e7e.pdf} \\ 
  defect\_issues & All issues labeled as defect. & 0.00 & 5.51 & 0.00 & 12.00 & \includegraphics[scale = 0.1, clip = true, trim= 50px 60px 50px 60px]{hist-5d0ab95cd3e300c21b0a40d6bec58e4c.pdf} \\ 
  defect\_with\_exception & All issues labeled labeled as defect and including an exception. & 0.00 & 1.82 & 0.00 & 1.00 & \includegraphics[scale = 0.1, clip = true, trim= 50px 60px 50px 60px]{hist-502028d1a7e2e77414e2f9f63e0bcd50.pdf} \\ 
  defect\_with\_stack & All issues labeled labeled as defect and including a stacktrace & 0.00 & 1.73 & 0.00 & 1.00 & \includegraphics[scale = 0.1, clip = true, trim= 50px 60px 50px 60px]{hist-dab0764147debd99066c84814d46d50e.pdf} \\ 
   \hline
\end{tabular}
\caption{Descriptive statistics for the analyzed issue dataset. Historgrams are in log scale.} 
\label{tab:stacktrace-stats}
\end{table*}


\begin{table}
\centering
\begin{tabular}{p{1em}lrr}
\hline
 \multicolumn{2}{l}{\bfseries{Java repository-level metrics}} & \\
\hline
\multicolumn{2}{l}{\bfseries{Repositories}} & 16,836\\
    & incl. exception on 1+ issues & 4,776 (28,37\%) \\
    & incl. stack trace on 1+ issues & 3,758 (22,32\%) \\
    & incl. exception on 1+ defect issues & 2,019 (41,44\%)\\
    & incl. stack trace on 1+ defect issues  & 1,698 (34,84\%) \\

\hline
\end{tabular}
\caption{Projects x Stack Traces}
\label{tab:proj}
\end{table}

\begin{table}
\centering
\begin{tabular}{p{1em}lrr}
\hline
\multicolumn{2}{l}{\bfseries{Issue-level metrics}} & \bfseries{Java} & \bfseries{Android Subset}\\
\hline
   \multicolumn{2}{l}{\bfseries{Repositories}} & 16,836 & 482\\
 \multicolumn{2}{l}{\bfseries{Issues}} & 356,057 & 32,582\\
    & incl. exceptions & 30,236 (8,5\%) & 3,956 (12,1\%) \\
    & incl. stack traces & 21,013 (5,9\%) & 3,101 (7,3\%) \\
 \multicolumn{2}{l}{\bfseries{Defect Issues}} & 55,226 & 2,376 \\
    & incl. exception & 6,741 (12,2\%) & 1,557 (65,6\%) \\
    & incl. stack trace &  5,196 (9,4\%) & 1,393 (58,7\%)\\
\multicolumn{2}{l}{\bfseries{Exception Stack Traces}} &  & \\
    & on issues  & 28,800 & 4,208  \\
    & on issues labeled as defect & 6,529 & 1,699 \\
\hline
\end{tabular}
\caption{Quantitative overview of the analyzed dataset}
\label{tab:overview}
\end{table}

The use of the Github issue tracker in repositories hosted on Github is optional
and not widespread. As of March 2014, from the 332,864 non-fork Java
repositories registered in the GHTorrent dataset, only 44,323 have received an
issue report in their lifetime while from those only 16,837 are being watched by
external users. To ensure that our work targets projects that are openly used by
users other than their developers, we focused our analysis on just those
projects. Using the heuristics presented in Section~\ref{sec:android}, we
selected 482 repositories featuring Android projects. Table~\ref{tab:proj} and
Table~\ref{tab:overview} presents two views of the extracted dataset, namely
grouped by repository and grouped by issue.

From Table~\ref{tab:overview}, we can see that repositories that include at
least one exception or stack trace on an issue are fairly common: 28\% and 22\%
respectively. Considering only the exceptions and stack traces reported on
defect issues we find out that 41\% of the repositories reported at least one
exception in a defect issue, and approximately 35\% of the projects mentioned
one stack trace in at least one defect issue.

At the individual issue level, 8,5\% and 5,9\% of all issues reported in Java
repositories contain the full name of an exception or a stack trace
respectively. The numbers are similar if we isolate Android projects from the
general population (12,1\% and 7,3\%). Considering only the defect-issues
however, in the Android repositories as almost 58,7\% of defect-issues include a
stack trace. This is not true for the general case however.

% latex table generated in R 2.15.3 by xtable 1.7-1 package
% Tue May  6 11:34:13 2014
\begin{table*}[ht]
\centering
\begin{tabular}{rp{16em}rrrrc}
  \hline
Feature & Description & quant\_5 & mean & median & quant\_95 & histogram \\ 
  \hline
all & All issues recorded for the project. & 2.00 & 67.70 & 20.00 & 248.00 & \includegraphics[scale = 0.1, clip = true, trim= 50px 60px 50px 60px]{hist-5d4a87daa44c63097df9bb1f3506d541.pdf} \\ 
  with\_exception & All issues featuring an exception. & 1.00 & 8.16 & 2.00 & 34.00 & \includegraphics[scale = 0.1, clip = true, trim= 50px 60px 50px 60px]{hist-bf73235d0503f937ef6f36c28cf1ba0d.pdf} \\ 
  with\_stack & All issues featuring a stack trace. & 1.00 & 6.44 & 2.00 & 27.00 & \includegraphics[scale = 0.1, clip = true, trim= 50px 60px 50px 60px]{hist-528cb6be67074e6e2c22d11245920e13.pdf} \\ 
  defect\_issues & All issues labeled as defect. & 0.00 & 13.28 & 1.00 & 52.00 & \includegraphics[scale = 0.1, clip = true, trim= 50px 60px 50px 60px]{hist-eb1797e86ef23b2a07907e5ccbeff7b9.pdf} \\ 
  defect\_with\_exception & All issues labeled labeled as defect and including an exception. & 0.00 & 3.25 & 0.00 & 13.00 & \includegraphics[scale = 0.1, clip = true, trim= 50px 60px 50px 60px]{hist-f1a879616aa3e0995db35c6c02bc8520.pdf} \\ 
  defect\_with\_stack & All issues labeled labeled as defect and including a stacktrace & 0.00 & 2.92 & 0.00 & 10.00 & \includegraphics[scale = 0.1, clip = true, trim= 50px 60px 50px 60px]{hist-49dbfa02c366cab8b489f33014de587d.pdf} \\ 
   \hline
\end{tabular}
\caption{Descriptive statistics for the analyzed issue dataset for Android projects. Historgrams are in log scale.} 
\label{tab:android-stack-stats}
\end{table*}


%Based on the assumption presented in Section III, we therefore did not use this label for further filtering.

\noindent \fbox{
\begin{minipage}{0.96\columnwidth} \emph{ RQ1: 28,37\% of Java
  repositories on Github have at least 1 issue that includes a stack trace.
  9,4\% of the defect issues of such repositories mentioned at least one stack
  trace. Considering only defect-issues of Android subset 58,7\% mentioned at
  least 1 stack trace.  }
\end{minipage}}

\section{RQ2: What are the common characteristics of such stack traces?}

To answer this question, we mined from every extracted stack trace the top level
exception being signaled, its signaler (the first method signature frame after
the exception), the root exception (the exception which initiated the stack
trace, defined on the bottom of the stack), and the root signaler. The
intermediate elements that compose a stack trace as shown in
Figure~\ref{fig:wrapping} (i.e., the intermediate causes and its execution
stacks) were not be considered in the analysis.
%~\todo{Move this to the previous section? IMO, It is better to keep all facts
%about how we did the research in one place}

Based on the assumption that regardless the issue type, every exception stack
trace contains relevant information concerning the exception structure of
projects analyzed we opted for not restricting the analysis only on defect
issues~\footnote{We conducted the same analysis on the defect issues and the top
exceptions are similar to the ones mentioned on defect issues. Due to space
limitation we limit to present the general analysis here. More detailed analysis
on the defect issues can be found here:
\url{www.dimap.ufrn.br/~roberta/icsm2014}}.

\subsection{Top 10 most common exceptions in Java Projects}

Table~\ref{tab:toptenjava} illustrates the top 10 most common root causes which
represents more than 50\% of the analyzed set. The table illustrates  i) the
number of times each exception appeared as the root cause of a stack trace
(\#occurrences), ii) the number of distinct projects reporting such causes on
stack traces, and iii) the exception popularity among projects.

\begin{table}
\begin{tabular}{lrrrr}
    \hline
    \bfseries{Exception} &  \multicolumn{2}{c}{\bfseries{Occurrences}} &  \multicolumn{2}{c}{\bfseries{Projects}}\\
    & \bfseries{\#} &  \bfseries{\%} & \bfseries{\# } & \bfseries{\% } \\
    \hline
java.lang.NullPointerException             & 7578 & 26,31\% & 1671 & 44,47 \\
java.lang.IllegalArgumentException         & 1573 & 5,46\%  & 670  & 17,83\% \\
java.lang.ClassNotFoundException           & 1203 & 4,18\%  & 514  & 13,68\% \\
java.lang.RuntimeException                 & 1093 & 3,80\%  & 411  & 10,94\% \\
java.lang.IllegalStateException            & 964  & 3,35\%  & 431  & 11,47\% \\
java.lang.ClassCastException               & 886  & 3,08\%  & 408  & 10,86\% \\
java.lang.ArrayIndexOutOfBoundsException   & 834  & 2,90\%  & 345  & 9,18\% \\
java.lang.NoSuchMethodError                & 822  & 2,85\%  & 369  & 9,82\% \\
java.io.IOException                        & 656  & 2,28\%  & 337  & 8,97\% \\
java.lang.OutOfMemoryError                 & 601  & 2,09\%  & 235  & 6,25\% \\
%java.lang.NoClassDefFoundError            & 554  & 1,92\%  & 299  & 7,96\% \\
%java.util.ConcurrentModificationException & 465  & 1,61\%  & 180  & 4,79\% \\
%java.lang.IndexOutOfBoundsException       & 445  & 1,55\%  & 205  & 5,46\% \\
\hline
All	 &	23956 &	100\%	& 3758 & 100\%	 \\
\hline
  \end{tabular}
\caption{Top 10 most popular exceptions on issues in Java projects}
\label{tab:toptenjava}
\end{table}

We can observe that the NullPointerException was the one with the higher number
of occurrences (26,3\% of all stacks were caused by NullPointerExceptions) and
the one which had the highest popularity across projects. Actually, most of the
exceptions in the top 10 list (except java.lang.IllegalArgumentException,
java.io.IOException and java.lang.RuntimeException) are exceptions implicitly
thrown by Java Virtual Machine, when an expression violates the normal
semantics of the Java programming language (e.g. ArrayIndexOutOfBounds) or due
to an internal error or resource limitation (OutOfMemoryError).

The high prevalence of NullPointerExceptions, and the other implicitly-thrown
exceptions  is aligned with the findings other works~\cite{kim2013predicting,fraser20131600,csallner2004jcrasher}. For instance, Sunghun et
al.~\cite{kim2013predicting} showed that in Eclipse bug report system 38\% the bugs related to exception handling were caused by NullPointerException; other works on robustness testing~\cite{maji2012empirical,csallner2004jcrasher}
showed that most of the automatically
detected bugs were due to NullPointerExceptions and implicitly-signaled of Java
environment (as the ones found in this study).

\subsection{Top 10 most common exceptions in Android Projects}

As detailed in Section~\ref{sec:android}, 482 Android projects were carefully
selected to enable a deeper investigation of stack trace information, by
leveraging stack trace information with the source code and bytecode analysis.
For this subset of Android project, we could extract every package related to
each exception signaler, and based on this information we could classify every
stack trace based on its root signaler as presented in Table~\ref{tab:signalers}.

\begin{table}
  \centering
  \begin{tabular}{rp{22em}}
    \hline
    \bfseries{Type of signaler} & \bfseries{Description} \\
    \hline
    \bfseries{android} & When the exception is explicitly thrown by a method defined in Android Platform, in a the JDK library used by it.\\
    \bfseries{app}     & If the exception signaler was explicitly thrown by an application method or in a the JDK library used by it.\\
    \bfseries{libcore} & If the application was signaled by one of the libraries reused by Android Platform (i.e., org.apache.harmony, org.w3c.dom, sun.misc, org.apache.http, org.json, org.xml) besides the JDK. \\
    \bfseries{lib}     & If the exceptions was not defined by any of the elements above.\\
    \bfseries{java}    & If the application signaler and all the methods that composes the exception are defined in a the JDK library method.\\
    \hline
  \end{tabular}
  \caption{Sources of exceptions in Android}
  \label{tab:signalers}
\end{table}

For each exception we could discover the number of times it occurred on stacks,
the number of distinct projects such stacks were defined (and consequently its
popularity) and the of times it was signaled by different signaler types. We
could calculate the popularity of each root cause, considering the number of
distinct projects they were found and the whole set of projects analyzed with at
least one stack trace (482). Table~\ref{tab:toptenandroid} and
Figure~\ref{fig:androidsignaler} presents the mined data.

\begin{table*}
  \centering
  \begin{tabular}{rccccccccc}
    \hline
    \bfseries{Root Exception} &  \multicolumn{2}{c}{\bfseries{Occurrences}} &  \multicolumn{2}{c}{\bfseries{Projects}} & \textsf{android} & \textsf{libcore} & \textsf{app} & \textsf{lib} & \textsf{java} \\
    & \bfseries{\#} &  \bfseries{\%} & \bfseries{\# } & \bfseries{\% } &&&&&\\
    \hline
java.lang.NullPointerException            & 1225 & 30,08\% & 254 & 52,70\% & 473 & 18 & 595 & 137 & 2 \\
java.lang.IllegalStateException           & 234  & 5,75\%  & 99  & 20,54\% & 165 & 12 & 36  & 20  & 1 \\
java.lang.IllegalArgumentException        & 255  & 6,26\%  & 94  & 19,50\% & 146 & 6  & 64  & 39  & 0 \\
java.lang.RuntimeException                & 232  & 5,70\%  & 91  & 18,88\% & 167 & 1  & 47  & 17  & 0 \\
java.lang.OutOfMemoryError                & 180  & 4,42\%  & 56  & 11,62\% & 121 & 15 & 17  & 23  & 4 \\
java.lang.NoClassDefFoundError            & 73   & 1,79\%  & 52  & 10,79\% & 9   & 0  & 37  & 26  & 1 \\
java.lang.ClassCastException              & 94   & 2,31\%  & 49  & 10,17\% & 43  & 0  & 40  & 11  & 0 \\
java.lang.IndexOutOfBoundsException       & 127  & 3,12\%  & 47  & 9,75\%  & 47  & 0  & 71  & 8   & 1 \\
java.lang.NoSuchMethodError               & 57   & 1,40\%  & 40  & 8,30\%  & 9   & 0  & 39  & 9   & 0 \\
java.util.ConcurrentModificationException & 54   & 1,33\%  & 38  & 7,88\%  & 5   & 0  & 43  & 6   & 0 \\

    \hline
  \end{tabular}
\caption{Root Exceptions occurrences and popularity.}
\label{tab:toptenandroid}
\end{table*}

\begin{figure}
\centering
\includegraphics[width=\hsize]{top_exceptios_android.pdf}
\caption{Top exceptions in Android repositories.}
\label{fig:androidsignaler}
\end{figure}

Coding errors are the most common root causes for most types of signalers. We
could observe that the NullPointerException is still the exception with higher
number of occurrences. The NullPoiterExceptions are mainly signaled inside
Android platform and inside application code, although we also find
NullPointerExceptions being signaled from third-party libraries. Regarding
reusable code, there is no consensus if it is a good or bad practice to re-throw
NullPointerException. Some prefer to encapsulate such an exception on
IllegalArgumentException, while others~\cite{bloch2008effective} argue that the
NullPointerException makes the cause of the problem explicit and hence 
should not be wrapped.

%commented only for submission:
We could observe in this study that the NullPointerException and other
other implicitly signaled exceptions represented most of the exceptions reported
on issues in both Java and Android subset. For such exceptions, which represent
programming bugs or resource limitations, there is usually no proper handling
besides presenting an error message to the user and restarting the application -
only high fault tolerant systems need to provide solutions to handle them.

\noindent \fbox{
\begin{minipage}{0.96\columnwidth} 
\emph{RQ2: Most exceptions are due to programming mistakes
or generally irrecoverable errors.\todo{Revise this}}
\end{minipage}}


\subsection{Classification of the main causes of top exceptions}

To better understand the main causes of such exceptions we evaluated semantics
related to the top 100 root causes reported on stack-issues of both Java dataset
and Android subset (which correspond to approximately 77\% of all exceptions
reported reported on Java stack-issues and 95\% of all exceptions reported on
Android stack-issues), and classified them according to the the categories
presented in Table~\ref{tophundrend}.


\begin{table}
  \centering
  \begin{tabular}{lrr}
    \hline
    \bfseries{Category} & \bfseries{Java} & \bfseries{Android Subset} \\
    \hline
      Programming logic (java.lang and util) & 12625  (41,10\%) & 2235 (55,64\%)\\
      Resources (IO)                         & 12440 (40,50\%)  & 727 (18,10\%) \\
      Security                               & 220  (0,72\%)    & 165 (4,11\%)\\
      Concurrency                            & 633 (2,06\%)     & 116  (2,89\%)\\
      Backward compatibility                 & 1580 (5,14\%)    & 219 (5,45\%) \\
      Reflection                             & 1413 (4,60\%)    & 91 (2,27\%)\\
      Specific (GUI,FRAMEWORK)               & 340 (1,11\%)     & 197 (4,90\%)\\
      General (Error, Exception, Runtime)    & 1468 (4,78\%)    & 267 (6,65\%)\\
    \hline
  \end{tabular}
  \caption{Characterization of the 100 root causes.}
  \label{tab:tophundrend}
\end{table}

As we can see from Table 4, the that programming errors are the causes of most
of the exception stack traces reported on issues. In the Android subset we could
find more issues related to Backward compatibility and security issues than on
the general data set.

\subsection{Sizes of Stack traces}

Concerning the size of the stacks (i.e. the number of frames composing them), we
could observe that  sometimes it exceeded 100 frames (158 stack frames).
Manually inspecting the corresponding execution trace was caused by one of the
two reasons: recursive calls; exceptions signaled by a method defined deep down
in a reused framework; exceptions successively wrapped. Regardless the reason,
in exception flows involving many methods, it is piratically impossible to
handle the exception a way different than existing the  application, because
there is hardly enough contextual information to perform any other recovery
action.

%TODO: Include a Histogram here.


%Considering only the packages of each exception we could observe that 81,28\%
%of all analyzed set of stack traces reported exceptions defined by java.lang
%(see Table XXX).

%\begin{table}
% \centering
%\begin{tabular}{lrrrr}
%    \hline
% \bfseries{rank} & \bfseries{package} & \bfseries{occurr} &  \bfseries{\%occur} & % \bfseries{projects} \\
%     \hline
%1 & java.lang         & 19471 & 81,28\% & 3080 \\
%2 & java.io           & 1222  & 5,10\%  & 509 \\
%3 & java.net          & 740   & 3,09\%  & 311 \\
%4 & java.util         & 689   & 2,88\%  & 287 \\
%5 & java.lang.reflect & 115   & 0,48\%  & 78 \\
%\hline
%  \end{tabular}
%\caption{Top 5 most popular packages associated to the root exceptions of stack traces}
%\label{tab:toptenpopular2}
%\end{table}

\section{RQ3.  What kinds of exception (mis)-use emerge from stack trace analysis? }

To answer this question we analyzed general characteristics of reported stack traces and, in the case of stack traces reported on Android repositories, we combine stack trace information with
source code and byte code analyses to enable a more detailed analysis.
Doing so, we mined some characteristics of stacks that points to best practices violations.

\subsection{General exceptions being thrown.}

From Tables~\ref{tab:toptenjava} and ~\ref{tab:toptenandroid} we can observe that,
in both Java and Android related issues, direct instances
java.lang.RuntimeException were thrown  in 10,94\% of Java repositories and
18,88\% of Android repositories respectively. Moreover, after inspecting the
top 100 most thrown exceptions (see ~\ref{tophundrend}) we could observe that
stack traces on which direct instances of java.lang.Exception and
java.lang.Error were thrown. 

Throwing general exceptions is usually considered a
bad practice, because the type does not carry enough information to identify the
cause of the exceptional behaviour. In such cases, developers need to relying on
exception message which can be neither complete nor precise
~\cite{gosling2000java}.

In Table ~\ref{toptenandroid} we can observe a high prevalence of
RuntimeExceptions being thrown by the Android platform (167 out of 232).
Manually inspecting the code on which RuntimeExceptions were thrown in Android
Platform we could observe that direct instances of java.lang.RuntimeException
they were thrown for 2 main reasons: (i) when there was no reasonable way of
handling the exception, which means that regardless the kind of exception the
only possible action was to abort the application; and when wrapping exceptions
signaled by application methods. Since Android platform is heavily based on
"call backs", when the application does not handle its exceptions, they scape to
Android platform which does not have contextual information to adequately handle
the exception and simply terminates the application. What is apparently a bad
practice of signaling general exceptions in this context it was and intended
design decision.

%Even an instance of Throwable as signaled in one project. It points to a the
%"dont worry from know behaviour".

%\subsection{Combining Stack Trace Information with Exception Type Information}


%todo{DISCOMMENT AFTER SUBMISSION}
%\subsection{Undocumented runtime exceptions signaled by third party code}

%In this study we also classified each stack trace according to the type of the
%root exception been signaled. Table~\ref{typeroottab} presents the types of
%root causes of all stack traces reported on the Android repositories.


%TO DO: INCLUDE A NEW TABLE WITH NON JAVA.LANG RUNTIME EXCEPTIONS. Point to we cannot trust developers to document.

%todo{discomment after submission} As we can see from Table ~ref{typeroottab},
%that most of the exceptions signaled by third-party code (e.g., Android and
%libs) were runtime exceptions. When an exception is signaled by third party
%code, it can be due to 2 reasons: the precondition associated to third party's
%code was not obeyed; or there is a coding error on third party's code. Only
%manual code inspection can reveal the reason. However, regardless the reason it
%is a good practice that all the exceptions signaled by a third part code are
%properly documented. Manually inspecting a sample of stack traces caused by
%runtime exceptions thrown by library code (approximately 10\%) we could observe
%that none of the runtime exceptions signaled by third part code were documented
%in Javadoc - this analysis did not consider the exceptions implicitly signaled
%by JVM (e.g. NullPoiterException). The undocumented runtime exceptions are a
%serious problem as it makes very difficult or even impossible for the  client
%method to protect against  undocumented runtime exceptions signaled by library
%code. As a consequence, the undocumented runtime exception may remain uncaught
%leading to  system crashes.



%This result is aligned with the results of other study conducted by from Cabral
%and Marcques ~\cite{} which also observe that runtime exceptions are most often
%not documented.

%\textbf{When libraries are used, the developer does not have access to their
%source code and thus needs to rely on the library documentation about the
%runtime exceptions that should be thrown – which, more often than not, are
%neither complete nor precise (Thomas, 2002; Sacramento et al., 2006). As a
%consequence, unhandled unchecked exceptions can be seen as one of the major
%sources of bugs in current Java systems (Jo et al., 2004). }

%On the other hand, we could observe that most of the exceptions signaled by the
%set of libraries reused by Android platform (i.e., org.apache.harmony,
%org.w3c.dom, sun.misc, org.apache.http, org.json, org.xml, javax.) referred in
%this work as libcore, were checked exceptions. It is pointed as a good practice
%for libraries (see Section ~\ref{best}) because by using checked exceptions
%libraries can define a precise exception interface to its clients. Such
%libraries are heavily used in several projects, and such precise exception
%interface may be related to the libraries maturity.

\subsection{Uncaught Checked Exceptions}

In this study we also classified each stack trace according to the type of the
root exception been signaled. Table~\ref{typeroottab} presents the types of root
causes of all stack traces reported on the Android repositories.

\begin{table}
\centering
\begin{tabular}{lcccccc}
    \hline
    \bfseries{Type} & \bfseries{Android} & \bfseries{Libcore} & \bfseries{App} & \bfseries{Lib} & \bfseries{Java} & \bfseries{All}\\
    \hline

Runtime   & 1101 & 56  & 1446 & 423 & 10 & 3036  \\
Error     & 148  & 34  & 168  & 114 & 9  & 473 \\
Checked   & 115  & 241 & 139  & 176 & 12 & 683 \\
Undefined & 1    & 0   & 6    & 8   & 0  & 15  \\
Throwable & 0    & 0   & 1    & 0   & 0  & 1 \\
    \hline
All       &	1365	&	331	&	1760	&	721	&	31	&	4208	\\
    \hline
  \end{tabular}
\caption{Types of Root Exceptions.}
  \label{fig:typeroottab}
\end{table}


\begin{figure}
\centering
\includegraphics[scale=0.3]{chart_exceptiontypes.pdf}
\caption{Distribution of Checked and Unchedked exceptions.}
\label{fig:typeroot}
\end{figure}

Although most of the reported were runtime, we could also find checked
exceptions as the root causes of stack traces reported on issues. An exception
can remain uncaught in two circumstances: if all methods on the execution trace
on which this exception flows explicitly specifies the exception, or if the
checked exception is wrapped in a runtime and than re-thrown.Since checked
exceptions should be used to represent recoverable conditions, ignoring a
checked exception is considered a bad practice. When the developer is confronted
with a checked exception, the designer of the API is telling him to handle the
exceptional condition. To better understand what was causing checked exceptions
to scape from being handled we investigated the kinds of wrappings that happened
on stacks the stacks.  Next sections presents the wrappings found and discuss
about them.

%\noindent\emph{3) Odd Wrappings}
\subsection{Odd Wrappings}

Table~\ref{wrappingandroid} presents the wrappings found in this study for all
stack traces found in Android repositories. As we can see, some of the checked
exceptions where indeed wrapped in runtime exceptions or even errors, while most
of them were not wrapped along the stack trace (represented as the table lines
on which the root cause is none). This analysis also revealed unexpected
wrappings such as: Error wrapping a Checked exception; a Runtime wrapping and
Error; and an Error wrapping a Runtime.

\begin{table}
\centering
\begin{tabular}{llll}
    \hline
    \bfseries{Exception Type} & \bfseries{Root Cause} & \bfseries{Occurrences} & \bfseries{Projects} \\
    \hline
    Runtime   & None    & 2360 & 359 \\
    Runtime   & Runtime & 560  & 182 \\
    Checked   & None    & 422  & 117 \\
    Error     & None    & 381  & 141 \\
    Runtime   & Checked & 109  & 70  \\
    Checked   & Checked & 98   & 42  \\
    Runtime   & Error   & 55   & 38  \\
    Checked   & Runtime & 22   & 12  \\
    Error     & Error   & 15   & 14  \\
    Undefined & None    & 15   & 10  \\
    Error     & Runtime & 13   & 6 \\
    Checked   & Error   & 8    & 7 \\
    Error     & Checked & 7    & 4 \\
    Throwable & Runtime & 3    & 1 \\
    Undefined & Runtime & 3    & 3 \\
    Throwable & None    & 1    & 1 \\
    Undefined & Error   & 1    & 1 \\
    \hline
  \end{tabular}
\caption{Kinds of wrappings found on the stacks of 482 ANDROID projects found on Github}
\label{fig:wrappingandroid}
\end{table}

Java is the only language that provides a hybrid exception handling mechanism
which offers different types to represent different exception behaviors (see
Section ~\ref{extypes}) . According to Java specification Errors should not be
handled inside the system since they usually represent unrecoverable conditions
detected by the JVM such as OutOfMemoryError. Checked exceptions should
represent recoverable conditions and  Runtime exceptions represent are usually
related to coding errors from which the developer is not expected to recover
from.

%Table X illustrated examples of such wrappings.
%TODO INCLUIR ESTA TABLEA

Since there is no way of enforcing such conventions during program development,
we could observe the stack trace analysis combined with types these three types
of exceptions were used interchangeably. This interchangeability makes the
behaviour of the exception handling code more complex and less reliable. For
instance, considering a situation were an OutOfMemoryError (a situation that
should not be handled) was wrapped in a checked exception (a situation that must
he handled), in this case this wrapping is telling to the caller that it should
handled.


\section{Threats to Validity}

\subsection{Internal Validity} We used a heuristics-based parser to mine
exceptions from issues.  Our parsing strategy was conservative by default; for
example, we only considered exception names using a fully qualified class name
as valid exception identifiers, while, in many cases, developers use the
exception name in issue description. Conservative parsing may minimize false
positives, which was our initial target, but also tends to increase false
negatives, which means that some cases may have not been identified as
exceptions or stack traces. Our limited manual inspection did not reveal such
cases.

\subsection{External Validity} The results presented here were based on mining
issues on Github through the GHTorrent dataset. While comprehensive and
extensive, GHTorrent is not an exact replica of Github, so several issues might
be left out. Due to the way data collection works with GHTorrent, for projects
that are relatively inactive, GHTorrent might have not collected a significant
proportion of their issues. We did not investigate the extend of this threat on
our sample.

\subsection{Construct Validity}
Parts of our analysis are based on the availability of stack traces on issues on
Github. In using this dataset, we make an underlying assumption: the
stack traces reported on issues are representative of real crashes in
the applications. The assumption is impossible to mitigate without access to
the full set of crash data per application. Services exist to collect all
crash data from applications, so access to such a dataset will allow for
a thorough replication of our analysis, perhaps at the individual application
level.

\section{Related Work}

In this section, we present works thaat are related to our own, distributed in
four categories: (i) works that use the information available on stack traces;
(ii) empirical studies on the usage of Java exceptions and its fault proneness;
(iii) tools to extract stack traces information from natural language artifacts
(i.e., issues and emails);  and (iv) empirical studies involving Android apps.

\textit{Analysis of Java Exception Usage.} Since the manual analysis of the Java
exception-flow can easily become infeasible Robillard and Murphy~\cite{Robil00}
developed a tool called Jex that analyzes the flow of exceptions in Java source
code. Based on java source code this tool performs dataflow analysis in order to
find the propagation paths of checked and unchecked exception types. After this
tool other tools have been proposed to support the static analysis of exception
flows ~\cite{fu2007exception} ~\cite{coelho2008assessing}. The main limitation
off all tools is the number of false positives inherent to static analysis
solutions, which can lead to a high number of exception flows, specially if
considering Java Environment exceptions and exceptions signaled from libraries.
This was the fist of some tools proposed to statically analyzed the exception
flow of exceptions. Cabral and Marques~\cite{cabral2007exception} analyzed the
exception handling code of 32 open-source systems, both for Java and .NET. The
goal of their study was to identify how exceptions were handled in different
categories of systems. They examined the exception handlers and the respective
actions taken on them. They observed that the action handlers were very simple
(e.g., logging and present a message to the user). In our work differs from
previous works since in our study we tried to identify bad practices from the
combined use of stack traces extracted from issues and bytecode and source code
analysis. Finally, Reimer and Srinivasan~\cite{reimer2003analyzing} listed a set
of bad practices on exception handling that hinder software maintainability and
robustness. Their differs from ours it was solely based on their own experience
on working with Java enterprise applications and focus on source code examples
of bad patterns on ways of handling exceptions such as exception swallowing, log
verbosity.

\textit{Analysis and Use of Stack Trace Information.} Some works have
investigated the usage of stack trace information to support bug classification
and clustering~\cite{wang2013improving, kim2011crash, dhaliwal2011classifying},
fault-proneness prediction models~\cite{kim2013predicting} and even automated
bug fixing~\cite{sinha2009fault} tools.Kim et al~\cite{kim2011crash} use of
crash graphs (an aggregated form of multiple stack traces available in crash
reports) to detect duplicate crash reports and to predict if a given crash will
be fixed. Dhaliwal et al~\cite{} analyzed crash reports collected for Mozilla
Firefox and proposed a crash grouping approach based on the Levenshtein
distance. This work showed that a crash grouping approach can reduce bug fixing
time in approximately 5\%. Wang et al~\cite{} proposes an approach to define
correlated crash types automatically. Moreover it also provides a bug
localization method to locate and rank files related to the bug described on a
stack trace.  Schroter et al~\cite{schroter2010stack} conducted an empirical
study on the usefulness of stack traces for bug fixing. The study used the
development data of Eclipse project and could observe that developers fixed the
bugs faster when failing stack traces were included on bug issues. This study is
aligned with the one performed by Bettenburg et al. ~\cite{bettenburg2008makes}
which evaluated what content could make a good bug report, and stack traces were
mentioned as the second most important content for developers - just after the
"steps to reproduce" information. Sinha et al~\cite{sinha2009fault} proposed an
approach for locating and repairing faults that are caused by JVM implicitly
signaled exceptions related to coding errors such as NullPointerExceptions. This
approach uses the information available on the stack trace to guide a dataflow
analysis, to locate the source statement responsible for the exception. Kim at
al.~\cite{kim2013predicting} proposed an approach to predict the crash-proneness
of methods based information extracted from stack traces and methods' bytecode
operations. They used InfoZilla to extracted stack trace information from the
bug reports of two open-source projects, i.e. AspectJ and Eclipse. They could
observe that most of the stack traces were related to NullPointerException and
other JMV implicitly thrown exceptions had the higher prevalence on the analyzed
set of stacks.

\textit{Extracting Stack Traces from Natural language artifacts.} Nowadays many
software vendors embed automatic crash reporting tools in their software
systems. Hence, whenever a software crashes this tool sends a detailed crash
report to its vendors. Moreover, there are also find third party software
solutions specialized in bug reporting, most of them for the increasing marked
of mobile apps~\cite{BugSe14,BugSn14,Googl14,Acra14}. Hence most of the works
described previously used the  information provided by crass reporting tools.
Two works have proposed tools extract stack traces from natural language
artifacts such as issues and emails: Infozilla ~\cite{bettenburg2008extracting}
is based on a set of regular expressions that extract an the set of frames
related to a stack trace, the main limitation of this solution is that it is not
able to extract stack traces embedded on verbose log files (i.e., on which we
can find log text mixed with exception frames). Alberto and Lanza
~\cite{bacchelli2012content}  propose a solution to recognize stack trace frames
from development emails and relate it to code artifacts (i.e. classes) mentioned
on the stack trace.

\textit{Empirical studies using Android apps.} Ruiz et al.~\cite{Ruiz12}
investigated the degree of reuse across applications in Android Market, the
study showed that almost 23\% of the classes inherited from a base class in the
Android API, and that 217 mobile apps were reused completely by another mobile
app. Pathak et al.~\cite{Patha11} analyzed bug reports and developers
discussions of Android platform and found out that approximately 20\% of
energy-related bugs in Android occurred after an OS update. McDonnell et
al.~\cite{McDon13} conducted a case study of the co-evolution behavior of
Android API and 10 dependent applications using the version history data found
in github. The study found that approximately 25\% of all methods in the client
code used the Android API, and that the methods reusing fast-evolving APIs were
more defect prone then others. Vásquez et al.~\cite{Linar13} analyzed
approximately 7K free Android apps and observed that the last successful apps
used Android APIs that were on average 300\% more change-prone than the APIs
used by the most successful apps. Pingyu and Elbaum~\cite{Zhang12} analyzed bug
reports of 5 Android applications an observed that 29\% had to do with poor
exceptional handling code.  Our work differs from the others as it aims at
distilling the information of bug reports describing uncaught exceptions created
for mobile applications in Github and Googlecode, in order to get a first view
of what is causing the crashes across applications available and what the
characteristics of the stack traces can tell us about the exception structure of
those applications. %Moreover this work\ldots.

\section{Conclusion}

In this work, we perform the first large scale analysis of Java stack traces and
pinpoint how they can reveal bad programming practices. We mined and analyzed
the stack traces embedded in all issues of Java projects available GitHub.
Overall, our research set includes 356,057 issues define at 16,836 projects
(from which 28,800 stack traces were extracted). From this set, the stack traces
of 482 Android projects were investigated in more detail, in combination with
source code and bytecode analysis. In this study some patterns of exception
(mis)-use were consistently detected such as: unexpected wrappings (e.g., Errors
being wrapped in checked exceptions) - revealing that Java hybrid exception
model is not fully used according to its purpose;  undocumented runtime
exceptions signaled by third party code - which makes almost impossible for
library clients to protect against such exceptions; and a high prevalence of
java.lang exceptions reported on issues - representing approximately 50\% of the
analyzed issues.


%Nowadays many software vendors embed automatic crash reporting tools in their
%software systems. Hence, whenever a software crashes this tool sends a detailed
%crash report to its vendors. Moreover, we can also find third party sofware
%solutions specialized in bug reporting for different kinds of systems specially
%for the increasing marked of mobile apps~\cite{BugSe14,BugSn14,Googl14,Acra14}.
%There is planty of information to be mined...


\section*{Acknowledgment} This work is partially supported by: CNPq - Proc.
484209/2013-2 and the NWO TestRoots project (639.022.314).

\bibliographystyle{IEEEtran}
\bibliography{android-stacks}

% that's all folks
\end{document}

